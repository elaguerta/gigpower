\section{Conclusion}

This paper considered optimization of three phase unbalanced distribution systems.  To do so, we derived a linearized model of distribution system power flow that maps real and reactive power injections into squared voltage magnitude differences.  This approximate model can be viewed as an extension of the \emph{LinDistFlow} \cite{baran1989optimal} linear model to unbalanced distribution systems.  The model itself, although approximate, was also used to give insight into an interesting phenomena regarding voltage rises on lightly loaded phases.  As was discussed in the Simulation Results section, a rise in voltage magnitude in one phase can now be attributed to the fact that off diagonal components of \eqref{eq:M}-\eqref{eq:N} have \emph{opposite} signs of the diagonal components, indicating that large voltage drops in some phases actually contribute to voltage \emph{rises} in others.

Using the linear approximate model, we also developed an Optimal Power Flow (OPF) program to drive system voltages to within a $\pm 5\%$ threshold of 1 p.u. while simultaneously minimizing the squared Euclidian distance between squared voltage magnitudes at each node. Our approach resulted in voltages that were much more balanced at each node, compared to the uncontrolled case.

Although the objective considered in this work was to balance feeder voltages, the derived model is capable of optimizing over a variety of objectives as the equality constraints are now \emph{linear}.  Our future work is aimed at exploring other useful applications of three phase unbalanced OPF such as voltage reference tracking, battery and electric vehicle charging, and, perhaps, forecasting.