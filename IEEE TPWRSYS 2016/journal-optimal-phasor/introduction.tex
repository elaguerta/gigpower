\section{Introduction}
\label{sec:introduction}

Coordination of a diverse set of Distributed Energy Resources (DER) presents many challenges to utility operators, who strive to ensure power of sufficient quality and quantity is available to retail customers at least cost.  Such assets can vary in size and nature from rooftop PV units and larger PV arrays, to battery storage systems, and demand response resources.  Lack of intelligent management and coordination of DER (distributed PV, specifically) has already resulted in financial impacts for customers and the utility in Hawaii \cite{stewart2013analysis}.  However, DER could, under the correct operational control scenarios, provide numerous benefits to the grid, including voltage support, ancillary services \cite{doe2015ADMS} and, perhaps, new sets of services that were previously infeasible due to lack of controlability or observability.

In distribution systems, it is important to distinguish control strategies based on balanced and unbalanced analysis.  Indeed, a variety of strategies for the management of DER presently exist in which decisions are based on knowledge of balanced distribution system models.  Turitsyn et al. \cite{turitsyn2011options} considered a suite of distributed control strategies for reactive power compensation using four quadrant inverters.  The work of \cite{li2014real} studies distributed voltage regulation in the absence of communication, relying on locally obtained information.  In \cite{robbins2013two}, a two-stage control architecture for voltage regulation is considered where distributed controllers inject power based on local sensitivity measurements.  The authors of \cite{zhang2013local} study local voltage reference tracking with integral-type controllers, based on local voltage measurements.  The authors of \cite{farivar2011inverter} address voltage regulation and loss minimization through solving an Optimal Power Flow (OPF) problem, and address convexity issues using second order cone relaxations.  The work of \cite{lam2012optimal} also considers an OPF approach for voltage regulation in distribution networks by framing the decision-making process as a semidefinite program.  The authors provide conditions under which the semidefinite relaxation of the non-convex power flow problem is tight in balanced circuits. It is worth noting that many of the aforementioned approaches can be traced back to the seminal work of \cite{baran1989optimal}, that introduced nonlinear and linear-approximated recursive branch power flow models.  Finally, several of the authors of this paper have considered solving balanced optimal power flow problems where decisions are made in the absence of a network model \cite{arnold2015model}.

However, as since neither loads nor impedances on all three phases are necessarily close to equal, and individual controllable DER may only be connected to single phases, strategies that consider individual phases as well as their mutual coupling effects need to be considered.  Approaches to coordinate DER in unbalanced distribution systems, while being critical to the practical application in real distribution systems and microgrids \cite{doe2015ADMS}, are much less prevalent in the literature.  
While there is consensus about the physics-based models \cite{kersting2012distribution}, using these in an optimization setting is challenging as the nonlinear nature of power flow equations introduces considerable complexity that can be prohibitive for optimal power flow (OPF) calculations. Perhaps the best known efforts have been put forth by Dall'Anese et. al \cite{dall2012optimization}, \cite{dall2013distributed}, who consider Semi-Definite Program (SDP) relaxations for OPF problems in unbalanced systems, but do not provide conditions under which feasibility and optimality are guaranteed.  In addition to inefficient scaling as the problem size grows, the work of \cite{bitar2014} points out that it becomes more difficult to find a tight relaxation as the ratio of constraints to network buses increases.  A likely reason that more strategies focusing on coordination of distributed energy resources in unbalanced systems do not exist is the lack of suitable linear models that approximate three phase power flow.

A key feature of linear models that makes them so attractive to incorporate into DER control strategies lies in their versatility to enable new types of problems to be formulated and solved.  In our previous works \cite{arnold2015optimal} \cite{sankur2016linear}, we proposed a linearized unbalanced power flow model that can be viewed as an extension of the \emph{LinDistFlow} \cite{baran1989optimal} linear approximation for balanced systems.  This model was incorporated into an OPF designed to balance voltage magnitudes in three phase systems.  Such an activity, to our knowledge, cannot be formulated as an SDP.

Another emerging application to which unbalanced linearized power flow models may be applied (to manage DER) is to enable fast and safe switching of circuit elements in distribution systems.  The ability to island/reconnect microgrids and reconfigure distribution feeders are seen as two important applications of future grids \cite{grid2015}, \cite{quad2015}.  Prior to opening or closing a switch, it is desireable to minimize the voltage phasor difference across the switch to ensure the distribution system is not overly disturbed by large instantenous power flows.  The ability to ``cleanly'' switch elements into and out of a given system could allow for faster restoration of electrical services to critical loads following a disaster, or allow for damaged components to be isolated for repair or replacement.  

While most typical distribution systems do not have the sensing equipment to monitor voltage \emph{phasors}, the growing presence of distribution Phasor Measurement Units (PMUs), indicates that sufficient infrastructure may be in place in future grids to support control activities seeking to regulate feeder voltage phasors.  In fact, a small, but growing, number of control applications that utilize phase angle measurements have started to appear in literature.  The work of \cite{ochoa2010angle} proposed the use of voltage angle measurements to curtail over-generation of renewables.  Additionally, the authors of \cite{wang2013pmu} considered voltage angle thresholds as criteria to connect renewable generation.  Both works refer to this control activity as ``Angle Constrained Active Management'', or ACAM.

In order to enable a control strategy that can regulate voltage phasors, in this work we extend the previously studied model \cite{arnold2015optimal}, \cite{sankur2016linear} which we refer to as the \emph{LinDist3Flow} system, to consider voltage phase angles, thereby allowing OPF formulations to manage voltage phasors rather as opposed to only magnitudes.  

The specific activity studied herein is an OPF formulation that minimizes the voltage phasor difference across an open switch in a distribution system while simultaneously regulating feeder voltage magnitudes to within acceptable limits.  In the event that one of the phasors is uncontrolled (i.e. a reference signal), then this activity can be though of as a voltage phasor tracking problem.  In driving the voltage phasor difference across a circuit element to 0, we ensure that when the switch is closed, only small amounts of power will flow across this element.  In this manner, the switch can be closed without disturbing the rest of the voltage profile in the feeder.  

We first discuss the \emph{Dist3Flow} equations and extend the system to model voltage angles in Section \ref{sec:analysis}.  Here, we also apply simplifying assumptions to the extended \emph{Dist3Flow} system to derive a linear model suitable for incorporation into an optimal power flow formulation (we refer to the linearized system as the \emph{LinDist3Flow} model).  Simulation results of an OPF that uses the \emph{LinDist3Flow} system to track a voltage phasor reference at a specific point in the network, and regulate system voltage magnitudes are presented in Section \ref{sec:simulation_results}.

% of 3-phase complex power flow and introduce linearizations. In \ref{subsec:mag_general}, a model of voltage magnitude on 3-phase unbalanced systems is presented, as are linearizing assumptions, and a special case is discussed in \ref{subsec:mag_nominal}. In \ref{subsec:angle_general}, we derive a nonlinear set of equations relating the phase angle to power injections on a 3-phase network, and discuss linearizing assumptions. A special case of the linearized system is discussed in \ref{subsec:angle_nominal}.