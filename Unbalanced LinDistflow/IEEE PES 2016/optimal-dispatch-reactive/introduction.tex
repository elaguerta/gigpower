\section{Introduction}

Coordination of a diverse set of Distributed Energy Resources (DER) presents many challenges to utility operators, who strive to ensure power of sufficient quality and quantity is available to retail customers at least cost.  Such assets can vary in size from residential rooftop PV units to larger PV arrays and, perhaps, battery storage systems located at residential, commercials and industrial sites.  As has been experienced in Hawaii~\cite{stewart2013analysis}, the negative impact of high levels of distributed PV under current interconnection standards is significant, and has led to financial impacts to both consumers and the utility.  However, distributed generation resources could, under the correct operational control scenarios, provide numerous benefits to the grid, including voltage support and VAR compensation.  

A variety of strategies for the management of DER presently exist for \emph{balanced} distribution system models.  Turitsyn et al. \cite{turitsyn2011options} considered a suite of distributed control strategies for reactive power compensation using four quadrant inverters.  The work of \cite{li2014real} studies distributed voltage regulation in the absence of communication, relying on locally obtained information.  Their results extend to optimize the cost of reactive power and are based on linearized power flow approximations.  In \cite{robbins2013two}, a two-stage control architecture for voltage regulation is considered where distributed controllers inject power based on local sensitivity measurements.  The authors of \cite{zhang2013local} study local voltage reference tracking with integral-type controllers, based on local voltage measurements.  The authors of \cite{farivar2011inverter} address voltage regulation and loss minimization through solving an Optimal Power Flow (OPF) problem, and address convexity issues using second order cone relaxations.  The work of \cite{lam2012optimal} also considers an OPF approach for voltage regulation in distribution networks by framing the decision-making process a semidefinite program.  The authors provide conditions under which the semidefinite relaxation of the nonconvex power flow problem is tight in balanced circuits. It is worth noting that many of the aforementioned approaches can be traced back to the seminal work of \cite{baran1989optimal}, that introduced nonlinear and linear-approximated recursive branch power flow models.

Approaches to coordinate DER in \emph{unbalanced} distribution systems are much less prevalent.  Perhaps the best known efforts have been put forth by the authors of \cite{dall2012optimization}, who consider semidefinite relaxations for OPF problems in unbalanced systems, but do not provide conditions under which optimality is guaranteed.  In addition to inefficient scaling as the problem size increases, the work of \cite{bitar2014} points out that it becomes more difficult to find a tight relaxation as the ratio of constraints to network buses increases.  A likely reason that more strategies focusing on coordination of distributed energy resources in unbalanced systems do not exist is the lack of suitable linear models that approximate three phase power flow. 

This work attempts to fill that void by proposing a linearized unbalanced power flow model that can be viewed as an extension of the \emph{LinDistFlow} \cite{baran1989optimal} linear approximation for balanced systems.  Using this linear model, we construct an OPF formulation that dispatches reactive power resources from controllable power electronic inverters to perform voltage balancing and regulation.  Our results show that OPF formulations that utilize the linearized three phase \emph{LinDistFlow} model results in a dispatch of inverter VAR resources that successfully drives system voltages into acceptable regimes and simultaneously balances voltages by reducing magnitude differences between phases.

This work is organized as follows. A model of nonlinear three phase unbalanced power flow is introduced in Section \ref{sec:preliminaries}. This model, used in simulations later in the paper, is also used in the derivation of the linear approximation for three phase power flow in Section \ref{sec:magnitude_approx}. In Section \ref{sec:simulation_results}, we formulate an OPF to perform voltage regulation and balancing, and discuss results from simulation.  Finally, concluding remarks are provided in Section \ref{sec:conclusion}.