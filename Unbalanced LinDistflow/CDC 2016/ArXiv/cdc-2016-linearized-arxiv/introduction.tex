\section{Introduction}

Coordination of a diverse set of Distributed Energy Resources (DER) presents many challenges to utility operators, who strive to ensure power of sufficient quality and quantity is available to retail customers at least cost.  Such assets can vary in size from residential rooftop PV units to larger PV arrays and, perhaps, battery storage systems located at residential, commercials and industrial sites.  As has been experienced in Hawaii~\cite{stewart2013analysis}, the negative impact of high levels of distributed PV under current interconnection standards is significant, and has led to financial impacts to both consumers and the utility.  However, distributed generation resources could, under the correct operational control scenarios, provide numerous benefits to the grid, including voltage support, VAR compensation, and ancillary services \cite{doe2015ADMS}.  

A variety of strategies for the management of DER presently exist for \emph{balanced} distribution system models.  Turitsyn et al. \cite{turitsyn2011options} considered a suite of distributed control strategies for reactive power compensation using four quadrant inverters.  The work of \cite{li2014real} studies distributed voltage regulation in the absence of communication, relying on locally obtained information.  In \cite{robbins2013two}, a two-stage control architecture for voltage regulation is considered where distributed controllers inject power based on local sensitivity measurements.  The authors of \cite{zhang2013local} study local voltage reference tracking with integral-type controllers, based on local voltage measurements.  The authors of \cite{farivar2011inverter} address voltage regulation and loss minimization through solving an Optimal Power Flow (OPF) problem, and address convexity issues using second order cone relaxations.  The work of \cite{lam2012optimal} also considers an OPF approach for voltage regulation in distribution networks by framing the decision-making process as a semidefinite program.  The authors provide conditions under which the semidefinite relaxation of the non-convex power flow problem is tight in balanced circuits. It is worth noting that many of the aforementioned approaches can be traced back to the seminal work of \cite{baran1989optimal}, that introduced nonlinear and linear-approximated recursive branch power flow models.

Approaches to coordinate DER in \emph{unbalanced} distribution systems, while being critical to the practical application in real distribution systems and microgrids \cite{doe2015ADMS}, are much less prevalent in the literature.  
While there is consensus about the physics-based models \cite{kersting2012distribution}, using these in an optimization setting is challenging. Perhaps the best known efforts have been put forth by Dall'Anese et. al \cite{dall2012optimization}, \cite{dall2013distributed}, who consider semidefinite relaxations for OPF problems in unbalanced systems, but do not provide conditions under which feasibility and optimality are guaranteed.  In addition to inefficient scaling as the problem size increases, the work of \cite{bitar2014} points out that it becomes more difficult to find a tight relaxation as the ratio of constraints to network buses increases.  A likely reason that more strategies focusing on coordination of distributed energy resources in unbalanced systems do not exist is the lack of suitable linear models that approximate three phase power flow. 

In our previous work \cite{arnold2015optimal}, we attempted to fill this void by proposing a linearized unbalanced power flow model that can be viewed as an extension of the \emph{LinDistFlow} \cite{baran1989optimal} linear approximation for balanced systems.  Using this linear model, we constructed an OPF formulation that dispatched reactive power resources from controllable DER to perform voltage balancing and regulation.  In this work, we extend the previously studied model; generalizing it to allow for other linearizations and comparing the results of OPF formulations that employ our linear model to those obtained via convex relaxations and semidefinite programming.

This work has three main contributions.  First, we derive a nonlinear model of unbalanced power flow in distribution systems which can be viewed as an extension of the \emph{DistFlow} system of equations, first studied in \cite{baran1989optimal}.  Secondly, we propose two approximations to linearize the model into a form suitable for incorporation into OPF formulations.  Finally, we evaluate the performance of the proposed linearizations numerically.  In one experiment, we demonstrate that an OPF driven by our linearized unbalanced power flow model produces results that closely approximate those obtained via an OPF that uses convex relaxations.  In a second experiment, we compare the proposed linearizations to perform voltage phase balancing and show that an iterative approach can be used to reduce linearization errors.

% Our results show that OPF formulations that utilize the linearized three phase \emph{LinDistFlow} model results in a dispatch of inverter VAR resources that successfully drives system voltages into acceptable regimes and simultaneously balances voltages by reducing magnitude differences between phases.

This work is organized as follows. A derivation of the three phase nonlinear power flow model is presented in Section \ref{sec:analysis}, along with two proposed linearization approaches.  Section \ref{sec:simulations} presents simulation results using the IEEE 13 node test feeder \cite{IEEEtestfeeder}.  Concluding remarks are provided in Section \ref{sec:conclusions}.  

% This section also introduces two linearizations for the derived model that render it suitable for incorporation into an OPF formulation.  In Section \ref{sec:simulation_results}, we formulate an OPF that uses one of the proposed linearization approaches to control DER to minimize feeder head real power consumption and compare these results to those obtained via solving a similiar OPF that uses convex relaxations and semidefinite programming.  Additionally, in this section, we formulate a second OPF designed to deploy DER resources to regulate and balance feeder voltages.  For this OPF, where it is not possible to formulate a representation as a semidefinite program, we compare the errors introduced by our proposed linearizations and discuss results from simulation.  Finally, concluding remarks are provided in Section \ref{sec:conclusion}.